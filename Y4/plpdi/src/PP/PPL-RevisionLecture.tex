\documentclass{article}

\usepackage{amssymb}
\usepackage{amsmath}
\usepackage{amsfonts}
\usepackage{xcolor}
\usepackage{bussproofs}
\EnableBpAbbreviations
\usepackage[a4paper, total={6in, 8in}]{geometry}
\hfuzz 40pt

\newcommand{\B}{\mathbb{B}}
\newcommand{\N}{\mathbb{N}}
\newcommand{\one}{\underline{1}\;}
\newcommand{\Nat}{\texttt{Nat}}
\newcommand{\add}{\texttt{add }}
\newcommand{\T}{\texttt{true }}
\newcommand{\F}{\texttt{false }}
\renewcommand{\succ}{\texttt{succ }}
\newcommand{\rarr}{\rightarrow}

\title{Principals of Programming Languages: Revision Lecture}
\author{Sam Barrett}


\begin{document}
\maketitle
\section{The Simply Typed $\lambda$-Calculus}
\subsection{Syntax}
The syntax of the Simply Typed $\lambda$-Calculus can be defined as:

$$
T ::= \mathbb{B} | T \rightarrow T 
$$
$$
M ::= x | \lambda x: T.M | M M | \texttt{true} | \texttt{false} | \texttt{if } M \texttt{ then } M \texttt{ else } M
$$

You can see here that we use the Church style for typing whereby, variables in $\lambda$ abstractions are annotated with types.

Values are atomic, i.e. they cannot be evaluated further and are of the form:

$$
V ::= \lambda x: T.M | \texttt{true} | \texttt{false}
$$

When we compute a term we are typically trying to reduce it to a value.

\subsection{Evaluation Contexts}
When we want to define the call-by-value small-step operational semantics of a language we use evaluation contexts. The Call-by-value evaluation contexts for the small-step operational semantics of $\lambda$-Calculus is defined as:

$$
C ::= \bullet | C M | V C | \texttt{if } C \texttt{ then } M \texttt{ else } M 
$$

A context is a term with a \textit{hole} ($\bullet$) in it. 

You can tell that this is the call-by-value evaluation context as you can see that we always evaluate the arguments of a application before the application itself.

These contexts yield the following rules:

$$
\frac{}{( \lambda x: T. M) V \rarr_v M[x\backslash V])}\beta
$$

$$
\frac{}{ \texttt{if true}  \texttt{ then } M \texttt{ else } N \rarr_v M } \texttt{IteT}
$$

$$
\frac{M \rarr_v N}{C[M] \rarr_v C[N]} \texttt{CTX}_C
$$

$$
\frac{}{\texttt{if false}  \texttt{ then } M  \texttt{ else } N\rarr_v N} \texttt{IteF}
$$
\subsection{Typing Rules}
And to facilitate the typing of these expressions we use the following typing rules:

$$
\frac{}{\Gamma,x : T \vdash x : T}  \texttt{VAR}
$$

$$
\frac{\Gamma, x : T \vdash M:U}{\Gamma \vdash \lambda x:T.M : T \rarr U} \texttt{ABS}
$$

$$
\frac{\Gamma \vdash M : T \rarr U \Gamma \vdash N :T}{\Gamma \vdash M N : U} \texttt{APP}
$$

$$
\frac{}{\Gamma \vdash \texttt{true} : \mathbb{B}} \texttt{T}
$$

$$
\frac{}{\Gamma \vdash \texttt{false} : \mathbb{B} } \texttt{F}
$$

$$
\frac{\Gamma \vdash M : \mathbb{B} \Gamma \vdash N :T \Gamma \vdash P :T}{\Gamma \vdash \texttt{if } M \texttt{ then } N \texttt{ else } P:T  } \texttt{ITE}
$$

\subsection{Church-Numerals}

We can define Church Numerals in the Simply Typed $\lambda$-Calculus as having the type $ \texttt{Nat} = (\B \rarr \B) \rarr \B \rarr \B $.
\begin{align*}
    \underline{0} &= \lambda f:\B \rarr\B. \lambda x: \B.x \\
    \underline{1} &= \lambda f: \B \rarr \B. \lambda x: \B. f x \\
    \underline{2} &= \lambda f: \B \rarr \B. \lambda x: \B. f (f x) \\
    \dots
\end{align*}

Essentially, the number is a counter of how many applications of $f$ appear.

We can now define a successor function, $ \texttt{succ}$ of type $ \texttt{Nat} \rarr \texttt{Nat}$ and an $ \texttt{add}$ function of type $ \Nat \rarr \Nat \rarr \Nat$:

\begin{align*}
    \texttt{succ} &= \lambda a : \Nat . \lambda f: \B \rarr \B. \lambda  x: \B. f( a f x) \\
    \texttt{add} &= \lambda a: \Nat. \lambda b: \Nat. \lambda f: \B \rarr \B . \lambda x: \B. a f (b f x)
\end{align*}

Our $ \texttt{add}$ function essentially concatenates the $f$s in $a$ with the $f$s in $b$ and our $ \texttt{succ}$ function appends an $f$ to the value $a$.

This encoding can be used to iterate over a function of type $\B \rarr \B$ by applying a function $f : \B \rarr \B$ to a base case $ x : \B$ $n : \Nat$ times.

$$ 
n f x 
$$

However, if one wants to iterate over a function of another type, say, $ T\rarr T$ one will need to define a new set of Church numerals over the type $T$, i.e. a type of the form $(T\rarr T) \rarr T \rarr T)$ (This is later solved through the use of System-F's paramerisation of types, similar to Monads)

\subsubsection{Exercise 1}

Can we iterate over a function of type $\Nat \rarr \Nat$ using this method? I.e. can we define $ \texttt{add} = \lambda a:\Nat. \lambda b:\Nat.a \texttt{ succ } b$?

For this to be possible, $a$ would need to have the type $(\Nat \rarr \Nat) \rarr (\Nat \rarr \Nat)$. However, our value $a$ is of the type $(\B \rarr \B) \rarr (\B \rarr \B)$. These types are not compatible, if we were to redefine $\Nat$ to fit this function we would end up with a recursive type definition which is not allowed within the Simply Typed $\lambda$-Calculus.

\textbf{Note: This is possible using System-F} 

\subsection{System-F}

We define the Church-style syntax of System-F as:

\begin{align*}
    T &::= \alpha | \B | \N | T \rarr T | \forall \alpha. T \\
    M &::= x | \lambda x:T.M | M M | \texttt{true} | \texttt{false} | \texttt{if } M \texttt{ then } M \texttt{ else } M | \\ & \texttt{let }  x = M \texttt{ in } M | \texttt{zero} | \texttt{succ} M | \texttt{pred} M | \texttt{iszero} M | \lambda \alpha .M | M\{T\}
\end{align*}

Here we have both Boolean $\B$ and Nat $\N$ ground types. This leads to simpler examples. 

System-F utilises a system of parameterised types. The general form of these types is $\forall \alpha .T$ This defines a family of types whereby for \textbf{any} type $ \alpha $. For example, given an expression $M: \forall \alpha.T$, we can construct a any type of the form $M : T[\alpha \backslash T']$ such as $M : T[\alpha \backslash \B]$ or $M : T[ \alpha \backslash \N]$

We also have what are known as \textit{type abstractions} in the form of $ \lambda \alpha.M$ and \textit{type applications}T of the form $M \{ T \}$

Our Values take the form:

$$
V ::= \lambda x:T.M | \texttt{true} | \texttt{false} | \texttt{zero} | \texttt{succ} M | \lambda \alpha .M
$$
We define our Call-by-value evaluation contexts as:

$$
C ::= \bullet | C M | V C | \texttt{if } C \texttt{ then }  M \texttt{ else }  M | \texttt{let } x = C \texttt{ in } M | \texttt{pred } C | \texttt{iszero } C | C \{ T \} 
$$

Which is the same as before with the extension of let, pred, iszero and type application.

We also have an extended set of Call-by-value small-step operational semantics rules:

\begin{prooftree}
    \AXC{}
    \RL{$\beta$}
    \UIC{$ ( \lambda x:T.M) V \rarr_v M[x\backslash V]$}
\end{prooftree}

\begin{prooftree}
    \AXC{$ M \rarr_v N$}
    \RL{$ \texttt{CTX}_C$}
    \UIC{$C[M] \rarr_v C[N]$}
\end{prooftree}

\begin{prooftree}
    \AXC{}
    \RL{$ \texttt{IteT}$}
    \UIC{$ \texttt{if true}  \texttt{ then } M \texttt{ else } N \rarr_v M $}
\end{prooftree}

\begin{prooftree}
    \AXC{}
    \RL{$ \texttt{IteF}$}
    \UIC{$ \texttt{if false}  \texttt{ then } M \texttt{ else } N \rarr_v N $}
\end{prooftree}

\begin{prooftree}
    \AXC{}
    \RL{$ \texttt{LetV}$}
    \UIC{$ \texttt{let} x = V \texttt{in} M \rarr_v M[x\backslash V]$}
\end{prooftree}

\begin{prooftree}
    \AXC{}
    \RL{$ \texttt{PredZ}$}
    \UIC{$ \texttt{pred zero} \rarr_v \texttt{zero}$}
\end{prooftree}

\begin{prooftree}
    \AXC{}
    \RL{$ \texttt{PredS}$}
\UIC{$ \texttt{pred} (\texttt{succ }M)  \rarr_v M$}
\end{prooftree}

\begin{prooftree}
    \AXC{}
    \RL{$ \texttt{IsZZ}$}
    \UIC{$ \texttt{iszero zero} \rarr_v \texttt{true}$}
\end{prooftree}

\begin{prooftree}
    \AXC{}
    \RL{$ \texttt{IsZS}$}
    \UIC{$ \texttt{iszero} ( \texttt{succ} M) \rarr_v \texttt{false}$}
\end{prooftree}


\begin{prooftree}
    \AXC{}
    \RL{$ \texttt{T}\beta$}
    \UIC{$ ( \lambda \alpha.M) \{ T \} \rarr_v M[\alpha \backslash T] $}
\end{prooftree}


And our typing rules are as follows:

\begin{prooftree}
    \AXC{}
    \RL{$ \texttt{VAR}$}
    \UIC{$\Gamma, x : T \vdash x : T$}
\end{prooftree}


\begin{prooftree}
    \AXC{$\Gamma, x :T \vdash M : U$}
    \RL{$ \texttt{ABS}$}
    \UIC{$\Gamma \vdash \lambda x: T.M : T \rarr U$}
\end{prooftree}

\begin{prooftree}
    \AXC{$\Gamma \vdash M :T \rarr U \; \Gamma \vdash N : T$}
    \RL{$ \texttt{APP}$}
    \UIC{$\Gamma \vdash M N : U$}
\end{prooftree}

\begin{prooftree}
    \AXC{}
    \RL{$ \texttt{T}$}
    \UIC{$\Gamma \vdash \texttt{true}: \B$}
\end{prooftree}

\begin{prooftree}
    \AXC{}
    \RL{$ \texttt{F}$}
    \UIC{$\Gamma \vdash \texttt{false}: \B$}
\end{prooftree}

\begin{prooftree}
    \AXC{$\Gamma \vdash M :\B \; \Gamma \vdash N:T \; \Gamma \vdash P:T$}
    \RL{$ \texttt{ITE}$}
    \UIC{$\Gamma \vdash \texttt{if } M \texttt{ then } N \texttt{ else } P : T $}
\end{prooftree}

\begin{prooftree}
    \AXC{}
    \RL{$ \texttt{Z}$}
    \UIC{$\Gamma \vdash \texttt{zero} : \N$}
\end{prooftree}

\begin{prooftree}
    \AXC{$ \Gamma \vdash M : \N$}
    \RL{$ \texttt{S}$}
    \UIC{$\Gamma \vdash \texttt{succ} M : \N$}
\end{prooftree}

\begin{prooftree}
    \AXC{$ \Gamma \vdash M : \N$}
    \RL{$ \texttt{P}$}
    \UIC{$\Gamma \vdash \texttt{pred} M : \N$}
\end{prooftree}

\begin{prooftree}
    \AXC{$ \Gamma \vdash M : \N$}
    \RL{$ \texttt{I}$}
    \UIC{$\Gamma \vdash \texttt{iszero} M : \B$}
\end{prooftree}

\begin{prooftree}
    \AXC{$\Gamma \vdash M : T \; \Gamma, x : T \vdash N :U$}
    \RL{$ \texttt{LET}$}
    \UIC{$\Gamma \vdash \texttt{let} x = M \texttt{ in } N : U$}
\end{prooftree}


\begin{prooftree}
    \AXC{$\Gamma \vdash M :T $}
    \RL{$ \texttt{TABS}$}
    \UIC{$\Gamma \vdash \lambda \alpha . M : \forall \alpha .T$}
\end{prooftree}

\textbf{Note TABS also requires that $ \alpha \notin \texttt{FV}(\Gamma)$ is satisfied} 
Where $ \texttt{FV}(\Gamma)$ is the set of free variables in $\Gamma$ i.e. $ \alpha $ should be a \textit{new} variable with respect to $ \Gamma$ in the hypothesis.

\begin{prooftree}
    \AXC{$\Gamma \vdash M :\forall \alpha . T $}
    \RL{$ \texttt{TAPP}$}
    \UIC{$\Gamma \vdash M \{ U \} : T[ \alpha \backslash U] $}
\end{prooftree}

\subsubsection{Examples}

We previously saw that in the Simply typed $\lambda$-Calculus we could not construct a function, $ \texttt{add}$ of type $ \Nat \rarr \Nat$. Using System-F this is possible through an abstracted definition of Church Numerals:

\begin{align*}
    \underline{0} &= \lambda \alpha . \lambda f: \alpha \rarr \alpha . \lambda x: \alpha .x  \\
    \underline{1} &= \lambda \alpha . \lambda f: \alpha \rarr \alpha . \lambda x: \alpha . f x  \\
    \underline{2} &= \lambda \alpha . \lambda f: \alpha \rarr \alpha . \lambda x: \alpha . f(f x ) \\
    \dots
\end{align*}

Again $ \texttt{succ}$ has type $ \Nat \rarr \Nat \rarr \Nat$ and $ \texttt{add}$ type $\Nat \rarr \Nat$:

\begin{align}
    \texttt{succ} &= \lambda a: \Nat. \lambda \alpha . \lambda f: \alpha \rarr \alpha. \lambda x: \alpha .f (a \{ \alpha  \} f x) \\
    \texttt{add} &= \lambda a: \Nat . \lambda b: \Nat . \lambda \alpha . \lambda f : \alpha \rarr \alpha . \lambda x: \alpha . a \{ \alpha  \} f ( b \{ \alpha  \} f x )
\end{align}

Now, given any numeral $n : \Nat$ we can iterate over a function $F$ of type $ T\rarr T$ as follows: $n \{ T \} F$ and we can define $ \texttt{add}$ more simply using $ \texttt{succ}$ as:

$$
\texttt{add} = \lambda a: \Nat. \lambda b: \Nat. a \{ \Nat \} \texttt{succ} b
$$

\underline{Exemplar 1}

Prove that $ \texttt{add} = \lambda a: \Nat. \lambda b:\Nat.a \{ \Nat \} \texttt{succ} b$ is well typed under System-F. Where $ \Gamma = a:\Nat, b:\Nat$ and assuming $ \texttt{succ}$ has type $\Nat \rarr \Nat$

\textbf{Remember: $\Nat$ is defined as $\forall \alpha ( \alpha \rarr \alpha) \rarr \alpha \rarr \alpha$}  

\begin{prooftree}
    \AXC{}
    \RL{$ \texttt{VAR}$}
    \UIC{$ \Gamma \vdash a : \Nat$}
    \RL{$ \texttt{TAPP}$}
    \UIC{$ \Gamma \vdash a \{ \Nat \} : (\Nat \rarr \Nat) \rarr \Nat \rarr \Nat$}
    \AXC{}
    \UIC{$ \Gamma \vdash \texttt{succ} : \Nat \rarr \Nat $}
    \RL{$ \texttt{APP}$}
    \BIC{$\Gamma \vdash a \{ \Nat \}  \texttt{succ} : \Nat \rarr \Nat$}

    \AXC{}
    \RL{$ \texttt{VAR}$}
    \UIC{$ \Gamma \vdash b : \Nat $}

    \RL{$ \texttt{APP}$}
    \BIC{$\Gamma, b:\Nat \vdash a \{ \Nat \} \texttt{succ} b : \Nat $}
    \RL{$\texttt{ABS}$}
    \UIC{$\Gamma, a:\Nat \vdash \lambda b:\Nat. a \{ \Nat \} \texttt{succ} b : \Nat \rarr \Nat $}
    \RL{$\texttt{ABS}$}
    \UIC{$\Gamma \vdash \lambda a:\Nat. \lambda b:\Nat.a \{ \Nat \} \texttt{succ} b : \Nat \rarr \Nat \rarr \Nat$} 
\end{prooftree}

\textbf{Remember: To proof something is well typed, construct a proof tree using the relevant typing rules and to prove something evaluates to a certain form use the context rules.} 

\underline{Exemplar 2}

What does $ \texttt{add } \underline{1} \; \underline{1}$ compute to?

\begin{prooftree}
    \AXC{}
    \RL{$\beta$} 
    \UIC{$ \add \one \rarr_v \lambda b:\Nat. \one \{ \Nat \} \succ b$}
    \RL{$ \texttt{CTX}_{\bullet \; \one}$}
    \UIC{$ \texttt{add } \one \one \rarr_v (\lambda b:\Nat. \one \{ \Nat \} \succ b )\one$}
\end{prooftree}

\begin{prooftree}
    \AXC{}
    \RL{$\beta$}
    \UIC{$(\lambda b:\Nat. \one \{ \Nat \} \succ b )\one \rarr_v \one \{ \Nat \} \succ \one$}
\end{prooftree}

\begin{prooftree}
    \AXC{}
    \RL{$ T\beta$}
    \UIC{$\one \{ \Nat \}  \rarr_v \lambda f: \Nat \rarr \Nat. \lambda x: \Nat. f x$}  
    \RL{$ \texttt{CTX}_{\bullet \succ \one}$}
    \UIC{$\one \{ \Nat \} \succ \one \rarr_v (\lambda f: \Nat \rarr \Nat. \lambda x: \Nat. f x) \succ \one $}
\end{prooftree}

\begin{prooftree}
    \AXC{}
    \RL{$\beta$}
    \UIC{$(\lambda f: \Nat \rarr \Nat. \lambda x: \Nat. f x) \succ \rarr_v \lambda x:\Nat. \succ x $}
    \RL{$ \texttt{CTX}_{\bullet \one}$}
    \UIC{$(\lambda f: \Nat \rarr \Nat. \lambda x: \Nat. f x) \succ \one \rarr_v (\lambda x:\Nat. \succ x)\one $}
\end{prooftree}

\begin{prooftree}
    \AXC{}
    \RL{$ \beta$}
    \UIC{$(\lambda x:\Nat. \succ x)\one \rarr_v \succ \one$}
\end{prooftree}

\begin{prooftree}
    \AXC{}
    \RL{$\beta$}
    \UIC{$ \succ \one \rarr_v \lambda \alpha. \lambda f: \alpha \rarr \alpha. \lambda x: \alpha. f(\one \{ \alpha  \} f x) $}
\end{prooftree}

This is our final stage. We cannot reduce a value any further. This value is not the same \textit{\underline{2}} that we defined earlier, but it is equivalent to it. I.e. we can define an equivalence relation between them.


\subsubsection{Inductively defined data structures under System-F}

We can define other data types in a similar way to how we have defined Church Numerals under System-F.

For example the list structure: 
\begin{itemize}
    \item $ \texttt{List} = \forall \alpha. ( \N \rarr \alpha \rarr \alpha) \rarr \alpha \rarr \alpha $
    \item $ \texttt{nil} : \texttt{List}$
    \item $ \texttt{nil} = \lambda \alpha. \lambda f: \N \rarr \alpha \rarr \alpha . \lambda x: \alpha .x$
    \item $ \texttt{cons} : \N \rarr \texttt{List} \rarr \texttt{List}$
    \item $\texttt{cons} = \lambda n: \N. \lambda l: \texttt{List}. \lambda \alpha. \lambda f: \N \rarr \alpha \rarr \alpha . \lambda x: \alpha. (f n)(l \{ \alpha  \} f x)$
\end{itemize}

Using this encoding, we can define a list of two elements $N_1$ and $N_2$ as:

$$
\text{List}[N_1,N_2]: \lambda \alpha. \lambda f: \N \rarr \alpha \rarr \alpha . \lambda x: \alpha. (f N_1) (f N_2 x)
$$

\subsection{Abstract Data Types}

We can extend System-F to capture abstract data types through the use of existential types.

The syntax of this extension can be defined as:

\begin{align*}
    T &::= \alpha | \B | \N | T \rarr T | \forall \alpha. T | T \times T | \exists \alpha. T \\
    M &::= x | \lambda x:T.M | M M | \T | \F | \texttt{if } M \texttt{ then } M \texttt{ else }  M \\
      & |\texttt{let } x = M \texttt{ in } M | \texttt{zero} | \succ M | \texttt{pred} M | \texttt{iszero} M \\ 
      & | \lambda \alpha .M | M \{ T \} \\
      & | \langle M,M \rangle | \texttt{fst} M | \texttt{snd} M \\
      & | \texttt{pack} \langle T,M \rangle  as T | \texttt{unpack} \alpha, x = M in M 
\end{align*}

Using existential ($\exists$) types we can \textit{hide} the implementation of a type.

We can say that an abstract data type is comprised of: 

\begin{itemize}
    \item An abstract name
    \item A concrete representation type
    \item A concrete implementation
    \item An abstract interface
\end{itemize}

Here the $ \alpha $ in the  existential type facilitates the abstract name.

Packs create abstract data types where $T$ is the concrete representation type, and M is the concrete implementation. 

We define an abstract data type in the form: 

$$
\texttt{pack} \langle T,M \rangle  \texttt{ as }\exists \alpha . U
$$


We extend the definition of values to include pairs and packs:

$$
V ::= \lambda x:T.M | \texttt{true} | \texttt{false} | \texttt{zero} | \texttt{succ} M | \lambda \alpha .M | \langle M,M \rangle | \texttt{pack} \langle T,M \rangle  \texttt{ as } T
$$

And our Call-by-value evaluation contexts are now: 

\begin{align*}
    C &::= \bullet | C M | V C | \texttt{if } C \texttt{ then }  M \texttt{ else }  M \\
      &| \texttt{let } x = C \texttt{ in } M | \texttt{pred } C | \texttt{iszero } C | C \{ T \}\\
      &| \texttt{unpack } \alpha ,x = C \texttt{ in } M | \texttt{fst } C | \texttt{snd }  
\end{align*}

We also extend our operational semantics rules:

\begin{prooftree}
    \AXC{}
    \RL{$\beta$}
    \UIC{$ ( \lambda x:T.M) V \rarr_v M[x\backslash V]$}
\end{prooftree}

\begin{prooftree}
    \AXC{$ M \rarr_v N$}
    \RL{$ \texttt{CTX}_C$}
    \UIC{$C[M] \rarr_v C[N]$}
\end{prooftree}

\begin{prooftree}
    \AXC{}
    \RL{$ \texttt{IteT}$}
    \UIC{$ \texttt{if true}  \texttt{ then } M \texttt{ else } N \rarr_v M $}
\end{prooftree}

\begin{prooftree}
    \AXC{}
    \RL{$ \texttt{IteF}$}
    \UIC{$ \texttt{if false}  \texttt{ then } M \texttt{ else } N \rarr_v N $}
\end{prooftree}

\begin{prooftree}
    \AXC{}
    \RL{$ \texttt{LetV}$}
    \UIC{$ \texttt{let} x = V \texttt{in} M \rarr_v M[x\backslash V]$}
\end{prooftree}

\begin{prooftree}
    \AXC{}
    \RL{$ \texttt{PredZ}$}
    \UIC{$ \texttt{pred zero} \rarr_v \texttt{zero}$}
\end{prooftree}

\begin{prooftree}
    \AXC{}
    \RL{$ \texttt{PredS}$}
\UIC{$ \texttt{pred} (\texttt{succ }M)  \rarr_v M$}
\end{prooftree}

\begin{prooftree}
    \AXC{}
    \RL{$ \texttt{IsZZ}$}
    \UIC{$ \texttt{iszero zero} \rarr_v \texttt{true}$}
\end{prooftree}

\begin{prooftree}
    \AXC{}
    \RL{$ \texttt{IsZS}$}
    \UIC{$ \texttt{iszero} ( \texttt{succ} M) \rarr_v \texttt{false}$}
\end{prooftree}


\begin{prooftree}
    \AXC{}
    \RL{$ \texttt{T}\beta$}
    \UIC{$ ( \lambda \alpha.M) \{ T \} \rarr_v M[\alpha \backslash T] $}
\end{prooftree}

\begin{prooftree}
    \AXC{}
    \RL{$ \texttt{FstP}$}
    \UIC{$ \texttt{fst } \langle M,N \rangle \rarr_v M$}
\end{prooftree}

\begin{prooftree}
    \AXC{}
    \RL{$ \texttt{SndP}$}
    \UIC{$ \texttt{snd} \langle M,N \rangle \rarr_v N$}
\end{prooftree}

\begin{prooftree}
    \AXC{}
    \RL{$ \texttt{UnP}$}
    \UIC{$ \texttt{unpack } \alpha ,x = \texttt{ pack } \langle T,M \rangle \texttt{ as } U \texttt{ in } N \rarr_v N[ \alpha \backslash T][x \backslash M] $}
\end{prooftree}

We extend our typing rules as follows:

\begin{prooftree}
    \AXC{}
    \RL{$ \texttt{VAR}$}
    \UIC{$\Gamma, x : T \vdash x : T$}
\end{prooftree}


\begin{prooftree}
    \AXC{$\Gamma, x :T \vdash M : U$}
    \RL{$ \texttt{ABS}$}
    \UIC{$\Gamma \vdash \lambda x: T.M : T \rarr U$}
\end{prooftree}

\begin{prooftree}
    \AXC{$\Gamma \vdash M :T \rarr U \; \Gamma \vdash N : T$}
    \RL{$ \texttt{APP}$}
    \UIC{$\Gamma \vdash M N : U$}
\end{prooftree}

\begin{prooftree}
    \AXC{}
    \RL{$ \texttt{T}$}
    \UIC{$\Gamma \vdash \texttt{true}: \B$}
\end{prooftree}

\begin{prooftree}
    \AXC{}
    \RL{$ \texttt{F}$}
    \UIC{$\Gamma \vdash \texttt{false}: \B$}
\end{prooftree}

\begin{prooftree}
    \AXC{$\Gamma \vdash M :\B \; \Gamma \vdash N:T \; \Gamma \vdash P:T$}
    \RL{$ \texttt{ITE}$}
    \UIC{$\Gamma \vdash \texttt{if } M \texttt{ then } N \texttt{ else } P : T $}
\end{prooftree}

\begin{prooftree}
    \AXC{}
    \RL{$ \texttt{Z}$}
    \UIC{$\Gamma \vdash \texttt{zero} : \N$}
\end{prooftree}

\begin{prooftree}
    \AXC{$ \Gamma \vdash M : \N$}
    \RL{$ \texttt{S}$}
    \UIC{$\Gamma \vdash \texttt{succ} M : \N$}
\end{prooftree}

\begin{prooftree}
    \AXC{$ \Gamma \vdash M : \N$}
    \RL{$ \texttt{P}$}
    \UIC{$\Gamma \vdash \texttt{pred} M : \N$}
\end{prooftree}

\begin{prooftree}
    \AXC{$ \Gamma \vdash M : \N$}
    \RL{$ \texttt{I}$}
    \UIC{$\Gamma \vdash \texttt{iszero} M : \B$}
\end{prooftree}

\begin{prooftree}
    \AXC{$\Gamma \vdash M : T \; \Gamma, x : T \vdash N :U$}
    \RL{$ \texttt{LET}$}
    \UIC{$\Gamma \vdash \texttt{let} x = M \texttt{ in } N : U$}
\end{prooftree}


\begin{prooftree}
    \AXC{$\Gamma \vdash M :T $}
    \RL{$ \texttt{TABS}$}
    \UIC{$\Gamma \vdash \lambda \alpha . M : \forall \alpha .T$}
\end{prooftree}

\textbf{Note TABS also requires that $ \alpha \notin \texttt{FV}(\Gamma)$ is satisfied} 
Where $ \texttt{FV}(\Gamma)$ is the set of free variables in $\Gamma$ i.e. $ \alpha $ should be a \textit{new} variable with respect to $ \Gamma$ in the hypothesis.

\begin{prooftree}
    \AXC{$\Gamma \vdash M :\forall \alpha . T $}
    \RL{$ \texttt{TAPP}$}
    \UIC{$\Gamma \vdash M \{ U \} : T[ \alpha \backslash U] $}
\end{prooftree}

\begin{prooftree}
    \AXC{$\Gamma \vdash M : T \; \Gamma \vdash N :U$}
    \RL{$ \texttt{Pair}$}
    \UIC{$\Gamma \vdash \langle M,N \rangle : T \times U$}
\end{prooftree}

\begin{prooftree}
    \AXC{$\Gamma \vdash M : T \times U$}
    \RL{$ \texttt{Fst}$}
    \UIC{$\Gamma \vdash \texttt{fst } M :T $}
\end{prooftree}

\begin{prooftree}
    \AXC{$\Gamma \vdash M : T \times U$}
    \RL{$ \texttt{Snd}$}
    \UIC{$\Gamma \vdash \texttt{snd } M :U $}
\end{prooftree}

\begin{prooftree}
    \AXC{$\Gamma \vdash M : U[ \alpha \backslash T]$}
    \RL{$ \texttt{Pack}$}
    \UIC{$\Gamma \vdash \texttt{pack } \langle T,M \rangle \texttt{ as } \exists \alpha .U : \exists \alpha .U$}
\end{prooftree}

\begin{prooftree}
    \AXC{$\Gamma \vdash M : \exists \alpha .T \; \Gamma, x : T \vdash N : U$}
    \RL{$ \texttt{Unpack}$}
    \UIC{$\Gamma \vdash \texttt{unpack } \alpha ,x = M in N : U$}
\end{prooftree}

\textbf{Note: $ \texttt{unpack}$ requires that the condition $ \alpha \notin \texttt{FV}(U)$} 

\subsubsection{Existential Types - Examples}

$$
\texttt{pack } \langle \B, \langle \T, \lambda x: \B. \texttt{if } x \texttt{ then }  \F \texttt{ else } \T   \rangle  \rangle  \texttt{ as } \exists \alpha . \alpha \times ( \alpha \rarr \alpha )
$$

For this example, the concrete representation type is Boolean ($\B$), we define two operations on this type. Firstly we define a constant of $\T$ and secondly we define an operation that takes a boolean and negates it. We pack this concrete implementation inside of the pack. Visible to the user is the name $ \alpha $ and the abstract types of the operations contained within the pack.

Another example could be the List type we defined earlier.
$$
\texttt{pack } \langle \texttt{List}, \langle \texttt{nil}, \texttt{cons} \rangle  \rangle \texttt{ as } \exists \alpha . \alpha \times (\N \rarr \alpha \rarr \alpha )
$$

Here the user does not know about the List type and instead is allowed to construct an $ \alpha $ using the first operation and to construct further structures using this initial $ \alpha $ ($ \texttt{nil}$) and the second operation that takes a $\N$ and a $ \alpha $.

\end{document}
