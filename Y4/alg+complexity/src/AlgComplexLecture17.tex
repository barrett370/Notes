\documentclass{article}
\title{Alternative Bases for Circuits}
\author{Sam Barrett}

\usepackage{amssymb,amsmath,graphicx,mdframed}

\newmdtheoremenv{definition}{Definition}
\newmdtheoremenv{theorem}{Theorem}
\newmdtheoremenv{proof}{Proof}
\newmdtheoremenv{lemma}{Lemma}

\begin{document}
\maketitle

\section{Alternative Bases}

So far we have looked a single (standard) basis which uses the gates $\{ \neg , \vee, \wedge \} $. However, we can design circuits using arbitrary gate types.

\begin{definition}
  A \textbf{basis} is a finite set \(\Omega = \{ f_{i} \}_{i<k}\) of Boolean functions (also referred to as \textit{connectives} ). We say that a basis \(\Omega\) is ((functionally) \textbf{complete} / \textbf{adequate})  if every Boolean function is computed by an expression or formula formed from variables and elements in \(\Omega\).
\end{definition}

\begin{definition}
  \(\Omega\)-Circuits.

  Let \(\Omega\) be a basis. An \(\Omega\)-circuit is defined just like a Boolean circuit except with non-source nodes labelled by arbitrary elements  $f : \{0,1 \}^{n} \rightarrow \{ 0,1 \} $ from \(\Omega\) with precisely $n$ incoming edges.
\end{definition}

The size and depth of an \(\Omega\) circuit is the same as for a Boolean circuit.
\begin{itemize}
  \item The \textbf{size} of an \(\Omega\)-Circuit: length of longest path from a source node to the \textit{sink}
  \item The \textbf{depth} of an \(\Omega\)-circuit: the number of nodes.
\end{itemize}

\section{The change-of-basis theorem}

\begin{theorem}
  Let \(\Omega_{1}\) and \(\Omega_{2}\) be two \textbf{complete } bases. For any \(\Omega_{1}\)-circuit $C_{1}$ there is an \(\Omega)_{2}\)-circuit $C_{2}$ computing a \textbf{logically equivalent} Boolean function with:

  \begin{itemize}
    \item $|C_{2}| = O(|C_{1}|)$
    \item $depth(C_{2}) = O(depth(C_{1}))$
  \end{itemize}
\end{theorem}

\begin{proof}
  (sketch)

  For each element $f_{i} \in \Omega_{1}$, let $F_{i}$ be an \(\Omega_{2}\) circuit of size $s_{i}$ and depth $d_{i}$ computing $f_{i}$.

  Let $s= \max_{i}s_{i}$ and $d=\max_{i}d_{i}$

  Construct $C_{2}$ by replacing each non-source node labelled $f_{i}$ by the \(\Omega_{2}\)-circuit $F_{i}$. We can now say that $|C_{2}| \leq s\cdot |C_{1}|$ and $depth(C_{2})\leq d \cdot depth(C_{1})$
\end{proof}

\section{\(\Omega\)-Rebalancing}

Say we have a complete basis \(\Delta = \{ 0,1,\delta \} \) containing the Boolean constants 0 and 1 along with the function $\delta : \{ 0,1 \}^{3} \rightarrow \{ 0,1 \} $ given by:

\[
  \delta(x,y,z) := (\neg x \wedge y) \vee (x \wedge z)
\]


\begin{theorem}
  Spira Balancing

  Let \(\Omega\) be a basis. For every \(\Omega\)-formula $A$ of size $n$, there is a logically equivalent \(\Delta\)-formula $A'$ s.t. $|A'| = n^{O(1)}$ and $depth(A') = O(\log n) = O(\log |A'|)$
\end{theorem}

We will show this for the case of \textbf{binary} complete bases containing $\{ 0,1 \} $:

\begin{lemma}\label{lemma:S}
  Let $T$ be a binary tree. There must exist some subtree $S$ of $T$ s.t.:

  \[
    \frac{1}{3}|T| < |S| \leq \frac{2}{3}|T|
  \]
\end{lemma}

\begin{proof}

  (sketch of Lemma~\ref{lemma:S})

  Starting from the root of a binary tree, repeatedly take the largest immediate subtree until one falls into the range specified in Lemma~\ref{lemma:S}. At each step the subtree \textbf{at most halves} in size, and so the first time the subtree has size $\leq \frac{2}{3}|T|$ it must also have size $> \frac{1}{3}|T|$

\end{proof}

\begin{proof}(Spira Balancing)

  We will continue by induction on $n$.

  Let $A$ be a \(\Omega\)-formula and let $B$ be a distinguished subformula of $A$ s.t. $\frac{1}{3}|A| < |B| \leq \frac{2}{3}|A|$ (exists by Lemma~\ref{lemma:S})

  We can say that $A = C(B)$, Where this notation means that $A$ is the combination of the subtree $B$ and the remaining tree $C$.  so $\therefore \frac{1}{3}|A| < |C(x)| \leq \frac{2}{3}|A|$ Meaning some amount in this range of $A$ exists in $C$ as $|B| + |C| = |A|$.

  We can now say that $A$ is logically equivalent to the $\Delta$-formula:

  We do this by taking a case analysis on the value of $B$ which can be 1 or 0 to form:

  \[
    \delta(B', C(0)', C(1)')
  \]

  Which means:

  \[
    \delta =
  \begin{cases}
    C(0)' & B = 0 \\
    C(1)' & B = 1 \\
  \end{cases}
\]


  Where $B', C(0)', C(1)'$ are obtained by the inductive hypothesis on $n$.

\end{proof}


\end{document}
