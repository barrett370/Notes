% Created 2021-02-08 Mon 14:40
% Intended LaTeX compiler: pdflatex
\documentclass[11pt]{article}
\usepackage[utf8]{inputenc}
\usepackage[T1]{fontenc}
\usepackage{graphicx}
\usepackage{grffile}
\usepackage{longtable}
\usepackage{wrapfig}
\usepackage{rotating}
\usepackage[normalem]{ulem}
\usepackage{amsmath}
\usepackage{textcomp}
\usepackage{amssymb}
\usepackage{capt-of}
\usepackage{hyperref}
\author{Sam Barrett}
\date{\today}
\title{Algorithms and Complexity: Online session week 2}
\hypersetup{
 pdfauthor={Sam Barrett},
 pdftitle={Algorithms and Complexity: Online session week 2},
 pdfkeywords={},
 pdfsubject={},
 pdfcreator={Emacs 27.1 (Org mode 9.5)}, 
 pdflang={English}}
\begin{document}

\maketitle

\section{Points on formative assignment}
\label{sec:org93630b0}

\begin{itemize}
\item Before trying to answer a question, make sure you understand it.
\item Try and unpack the question. Expand on all assumptions you are given.

\begin{enumerate}
\item Need to give a logspace TM that decides this language
We are not expected to fully describe the Turing machine and work out the precise space usage or running time. The amount of detail seen in the model answer.

The worse case \(\texttt{WS}_{M}(n) \leq c \cdot \log n\), this \textbf{\textbf{must}} hold for all bitstrings whether in the language or not.

\item Suppose \(L\) is a language in \(\bf{NP}\). Then we want to show \(L\) is in \(\texttt{PSPACE}\)
**Note: We must be sure to say \emph{'we want to show'} when talking about something we are yet to prove.

We have a polynomial \(p\) and a polytime machine \(M\) s.t. \(\forall x \in \{ 0,1\}^{*}, x \in L \Longleftrightarrow \exists u M(x,u) = 1\) where \(u\) is a bitstring whose length is \(p (| x |)\)

We want to show it is in \(\texttt{PSPACE}\), so we want to provide a TM \(N\) that decides \(L\) and uses polynomial space.

\uline{Algorithm to decide whether \(x \in L\):}

Step 1. work out \(p(|x|)\)

Step 2. go through each bitstring \(u\) of that length \& check whether \(M(x,u) = 1\)

Step 1: uses polynomial space.

Step 2: uses \(p(|x|)\) space to hold \(u\), it reuses this same space for each \(u\) that we test. \(M\) uses polynomial time, we know that anything that is polytime is polyspace!
Therefore, overall we must be using polynomial space.

\item We want \(f \notin O(g)\), we want \(f(n) > 2 g(n)\) infinitely often
\sout{-----}-----\sout{-----}-----\sout{-----}-----\sout{-----}
\begin{center}
\begin{tabular}{lrrrrrr}
n & 5 & 6 & 7 & 8 & 9 & 10\\
\end{tabular}
\end{center}
\sout{-----}-----\sout{-----}-----\sout{-----}-----\sout{-----}
\begin{center}
\begin{tabular}{lllllll}
Is \(f(n) \leq 2g(n)\)? & No & Yes & Yes & No & \ldots{} & \\
 &  &  &  &  &  & \\
\end{tabular}
\end{center}
\sout{-----}-----\sout{-----}-----\sout{-----}-----\sout{-----}
\begin{center}
\begin{tabular}{lllllll}
 &  &  &  &  &  & \\
\end{tabular}
\end{center}
\sout{-----}-----\sout{-----}-----\sout{-----}-----\sout{-----}
\end{enumerate}
\end{itemize}
\end{document}
