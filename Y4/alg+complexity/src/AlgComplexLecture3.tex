\documentclass{article}
\usepackage{amsmath, amssymb,amsfonts,tikz, mdframed}
\usetikzlibrary{shapes,arrows,calc,positioning,backgrounds}
\title{Algorithms \& Complexity: Lecture 3, Completeness and Reductions}
\author{Sam Barrett}

\newcommand{\T}{\texttt{T }}
\newcommand{\True}{\texttt{True }}
\newcommand{\F}{\texttt{F }}
\newcommand{\False}{\texttt{False }}
\newcommand{\NP}{\mathbf{NP}}
\renewcommand{\P}{\mathbf{P}}

\newmdtheoremenv{lemma}{Lemma}
\newmdtheoremenv{definition}{Definition}
\newmdtheoremenv{theorem}{Theorem}

\begin{document}
\maketitle

\section{SAT and its variants}
\label{sec:sat}

\subsection{Propositional connectives}
\label{subsec:proplogic}

A basic reminder of  Propositional logic and connectives:

\begin{itemize}
  \item \T (\True) and \F (\False) are the propositional \textbf{constants}
  \item $a \wedge b$ is \True if both $a$ and $b$ are \True, otherwise \False. \textbf{Conjunction}
  \item $a \vee b$ is \True if either $a$ or $b$ is \True, otherwise \False. \textbf{Disjunction}
  \item $\neg a$ is \True if $a$ is \False and vice versa. \textbf{Negation}
  \item $a \rightarrow b$ is \True if either $a$ is \False or $b$ is \True, but \False otherwise. \textbf{Implication}
\end{itemize}

\begin{lemma}\label{lemma:prop-logic}
  \textbf{A propositional expression can be evaluated in linear time.} \textit{This is done using the \textit{shunting yard} algorithm to translate into postfix notation and then evaluating using a stack.}
\end{lemma}

\subsection{Conjunctive normal form}
\label{subsec:cnf}

A formula is in CNF when it is a conjunction of disjunctions of variables and their negations.

For example,

\[
  (u_{0}\vee \bar{u_{1}} \vee u_{2}) \wedge (u_{1}\vee \bar{u_{2}} \vee u_{3}) \wedge \underbrace{(\bar{u_{0}\vee u_{2} \vee \bar{u_{3}}})}_{\text{clause}}
\]

Where in the above example $\bar{u}$ is the negation of $u$.

The disjunctions within the formula are called \textbf{clauses } and the variables are called \textbf{literals}

A clause can be written as $u \rightarrow (v \vee w \vee x)$ rather than $\bar{u} \vee v \vee w \vee x$

\subsubsection{3CNF formulae}
\label{subsubsec:3cnf}

A CNF formula is \textbf{3CNF} when each clause has at most 3 literals

\textbf{Note: any 3CNF clause can be written as an implication}

\subsubsection{Conversion to negation-free form}
\label{subsec:negneg}

A formula is negation free when there are no occurrences of $\neg$ or $\rightarrow $. However these are still permitted in the form of negated variables which are still literals.

Every formula is equivalent to one in negation-free form. Simply push in each negation to a literal using de Morgan's laws:

\begin{align*}
  \neg(\psi \vee \psi ') &= (\neg \psi) \wedge (\neg \psi ') \\
  \neg(\psi \wedge \psi ') &= (\neg \psi) \vee (\neg \psi ')
\end{align*}

Each push is $O(n)$-time, so the overall conversion is in $O(n^{2})$ time.

\subsubsection{Negation-free to CNF}
\label{subsec:neg-free-CNF}

A negation-free formula $\phi$ can be converted to a CNF formula $\phi '$ using extra free variables that are \textbf{equisatisfiable} with $\phi$. This means that the new formula will be satisfiable iff $\phi$ is satisfiable.

This is done via induction on $\phi$:

\begin{itemize}
  \item The case where $\phi$ is a literal is clear, literals are already in CNF
  \item The case where \(\phi\) is a conjunction is clear, it is already in CNF
  \item What if \(\phi\) is a disjunction?

        For any variable $c$ and clause \(\phi\), the formula $c \rightarrow \phi$ is equivalent to the clause $\bar{c} \vee \phi$.

        Therefore, any variable $c$ and CNF formula $\phi$, the formula $c \rightarrow \phi$ is equivalent to a CNF formula by the law:

        \[
        c \rightarrow (\psi\wedge\psi^{\prime}) = (c \rightarrow \psi) \wedge (c \rightarrow \psi^{\prime})
        \]

        For any CNF formulas \(\phi\) and \(\phi^{\prime}\), the formula \(\phi\vee\phi^{\prime}\) is equisatisfiable with:

        \[
        (c\vee c^{\prime}) \wedge (c \rightarrow \phi) \wedge (c^{\prime} \rightarrow \phi^{\prime})
        \]

        Which is equivalent to a CNF formula, obtained in $O(n)$ time. Thereby, we have a conversion to CNF in $O(n^{2})$


\end{itemize}

\subsubsection{CNF to 3CNF}
\label{subsec:CNF-3CNF}

In CNF, each clause is of the form:

\[
  a \rightarrow (b_{0} \vee \cdots \vee b_{n-1} \vee c \vee c^{\prime})
\]

is equisatisfiable with:

\begin{align*}
  &(a \rightarrow b_{0} \vee d_{0}) \\
  \wedge&(d_{0} \rightarrow b_{1} \vee d_{1}) \\
  \ldots &\\
  \wedge&(d_{n-2} \rightarrow b_{n-1} \vee d_{n-1}) \\
  \wedge&(d_{n-1} \rightarrow c \vee c^{\prime})
\end{align*}

Where $d_{0},\ldots, d_{n-1}$ are \textit{fresh} variables.

This gives an $O(n)$ time conversion to 3CNF


\subsection{Satisfiability}
\label{subsec:satisfiability}


Satisfiability is the process of answering questions of the form: \textit{Over the variables $p,q,r$ is the formula $(\neg(q \rightarrow p) \wedge r) \vee (p \wedge q)$}  satisfiable?

In this particular example, the answer is \textit{yes}, in the case where $p= \F$ and $q = r = \T$

\subsubsection{Formula-SAT}
\label{subsubsec:fsat}

\texttt{Formula-SAT} is the set of all formulas that are satisfiable.

\texttt{Formula-SAT} is in $\NP$, this is the case as given a formula $\phi$, and an interpretation $u$,
\begin{itemize}
  \item the length of $u$ is no longer than that of $\phi$
  \item it takes linear time to test whether it is a satisfying assignment by Lemma~\ref{lemma:prop-logic}.
\end{itemize}

\subsubsection{SAT}
\label{subsubsec:sat}

\texttt{SAT} is the set of CNF formulae that are satisfiable. Since \texttt{SAT} is a special case of \texttt{Formula-SAT} (which is in $\NP$), it too is in $\NP$

\subsubsection{3SAT}
\label{subsubsec:3sat}

\texttt{3SAT} is the set of 3CNF formulae that are satisfiable. Again, since it is a special case of \texttt{SAT}, it too is in $\NP$.

\section{Reductions}
\label{sec:reductions}

We often want to reduce a problem in mathematics/ Computer science to another, simpler or more understood problem. Intuitively, this can be thought of in the same way as reducing the problem of making \textit{profiteroles} to the problem(s) of making cream-filled pastries and making chocolate sauce.

Let $L$ and $L'$ be languages.

A (many-to-one) \textbf{reduction} from $L$ to $L '$ is a function $f : \{ 0,1 \} ^{*} \rightarrow \{ 0,1 \}^{*} $ such that for any bitstring, $x$ we have $x \in L$ iff $f(x) \in L '$.

Or, more plainly, if we know how to decide membership of $L '$, then the reduction enables us to decide membership of $L$.

\subsection{Computable reductions}
\label{subsec:comp-reductions}

We write $L \leq_{m} L '$ when there is a reduction from $L$ to $L '$ that is \textbf{computable}. From this we can see that:

\begin{itemize}
  \item If $L '$ is decidable, then $L$ is decidable
  \item If $L$ is undecidable (\textit{e.g. Halting problem}), then $L '$ is undecidable .
\end{itemize}

This is a very useful property and allows us to easily prove the deciability or undecidability of problems without explicitly having to prove them. We will \textbf{not} look any closer in this module.

\subsection{Polynomial time reductions}
\label{subsec:polytime-reductions}

We write $L \leq_{P } L '$ when there is a reduction from $L$ to $L ' $ that is \textbf{polynomial time}.
\begin{itemize}
  \item If $L '$ is in $\P$, then $L$ is also in $\P$
  \item If $L '$ is in $\NP$, then $L$ is also in $\NP$
\end{itemize}

\subsection{$\NP$-Completeness}
\label{subsec:np-complete}

A language $L$ is $\NP$-hard if \textbf{every} language in $\NP$ has a polynomial-time reduction to it.

Therefore, if $L$ is in $\P$ and $\NP$-hard then $\P = \NP$!

If $L$ is in $\NP$ and also $\NP$-hard, we say that it is $\NP$-complete. These are the \textit{hardest} problems in $\NP$.

\subsubsection{Proving $\NP$-completeness}
\label{ssec:provingNPComplete}

To prove that a problem is $\NP$-complete:

\begin{itemize}
  \item One must show that it is in $\NP$
  \item One must show that some other $\NP$-hard problem reduces to it.
\end{itemize}

\section{The Cook-Levin theorem}
\label{sec:cook-levin}

\begin{theorem}
  \label{theorem:cook-levin}
  3SAT is $\NP$-complete
\end{theorem}

We know that 3SAT is in $\NP$. Therefore, to show that it is $\NP$-complete, we must show it to also be in $\NP$-hard.

For any language $L \in \NP$ we want to give a polytime reduction from $L$ to 3SAT.

We will approach this in order from Formula-SAT $\rightarrow$ SAT $\rightarrow $ 3SAT

\subsection{Reducing to Formula-SAT}
\label{subsec:r-fsat}

Since $L$ is in $\NP$ there must be a nondeterministic Turing machine which decides it.

Say that $M$ is a NDTM for the language $L$, using an input tape, a work tape and an alphabet $\{ \rhd, \Box, 0,1 \} $ with 50 states and a running time and space usage of at most $n^{3}$, where $n$ is the size of the input.

From this, we must convert a bitstring $x$ of length $n$ into a propositional logic formula that is satisfiable iff $x\in L$.

\underline{The variables}
\begin{itemize}
  \item Let $a_{i,j,s}$ say that at time $i$, cell $j$ of the work tape contains symbol $s$. Here $i,j < n^{3}$
  \item Let $b_{i,j}$ say that, at time $i$ the input head is in position $j$. Here $i < n^{3}$ and $j < n$.
  \item Let $c_{i,j}$ say that, at time $i$, the work head is in position $j$. Here $i,j < n^{3}$
  \item Let $d_{i,q}$ say that, at time $i$ the current work state is $q$. here $i < n^{3}$ and $q < 50$ (as per machine definition)
\end{itemize}

\underline{The constraints}

\begin{itemize}
  \item For any time $i$, each cell $j$ contains only one symbol and there is only one current state.
  \item The configurations at time $i$ and time $i+1$, and the input, are related by the transition function.

        This is stated locally, meaning if, at time $i$ the state at time $i+1$ is determined only by adjacent states.
  \item At some time $i < n^{3}$, the current state is $q_{\texttt{accept} }$.
\end{itemize}

Putting these things together gives a formula of size $O(n^{3})$, It is satisfiable iff the bitstring $x$ is acceptable ($x \in L$).

\subsection{Reduction to SAT}
\label{subsec:red-sat}

By converting a formula to an equisatisfiable CNF formula in $O(n^{2})$ time (See Section~\ref{subsec:neg-free-CNF}), we show that Formula-SAT $\leq_{P}$ SAT.

\subsection{Reduction to 3SAT}
\label{subsec:red-3sat}


By converting a CNF formula to an equisatisfiable 3CNF formula in $O(n)$ time we show that SAT $\leq_{P}$ 3SAT

\subsection{Proving $\NP$-completeness}
\label{subsec:provingNPcomplete}

We previously outlined how to prove a problems is $\NP$-complete in Section~\ref{ssec:provingNPComplete}. We have shown that 3SAT reduces to $\NP$-hard thus satisfying the second point.


\section{Logspace reductions}
\label{sec:logspace}

We know that $\bf L \subseteq P$, i.e every decision problem that can be solved in logspace can be solved in polynomial time.

\subsection{Requirements}
\label{subsec:req}
\begin{itemize}
  \item We will write $L \leq_{L} L^{\prime}$ when there is a logspace reduction from $L$ to $L^{\prime}$.
  \item We \textit{want} the identity reduction to be logspace, i.e. $L \leq_{L} L$.
  \item We want a composite of logspace reductions to be logspace, so that:

        \[
        L\leq_{L}L^{\prime}\leq_{L}L^{\prime\prime} \implies L \leq_{L} L^{\prime\prime}
        \]

        \item If you have two languages that are related, $L \leq_{L} L^{'}$ then $L^{\prime} \in \mathbf{L} $ should imply $L \in \mathbf{L} $
  \item Every logspace reduction should be polytime, so that, $\leq_{L}$ implies $\leq_{P}$.
\end{itemize}


\subsection{Logspace reduction: definition}
\label{subsec:ls-definition}

We impose the following requirements on a function $f : \{ 0,1 \} ^{*} \rightarrow \{ 0,1 \} ^{*}$:

\begin{itemize}
  \item $f$ must be \textbf{polynomially bounded}, meaning, there must be a $c$ such that, for every bitstring $x$, $|f(x)| \leq |x|^{c}$ holds true.
  \item We can test in logarithmic space whether a particular position in the output it within or outside of the length of $f(x)$. Formally:

        The set of pairs $\langle x,i \rangle $, s.t. $i\leq |f(x)|$ \textbf{must} be in $\mathbf{L}  $
  \item We can test whether a particular position $\langle x,i \rangle $ gives a result of 1 or not, $f(x)_{i} = 1$ must be in $\bf L$. This is referred to as the \textbf{bitwise} problem for $f$. (This is the most important condition)
\end{itemize}

\subsection{Composing logspace reductions}
\label{subsec:ls-composing}

Suppose we have two logspace reductions $f$ and $g$, then the composite function $x\mapsto g(f(x))$ is also logspace, we are going to show that this is the case:

To compute $g(f(x))_{i}$ (the $i^{th}$ bit of the result) using three work tapes (A,B,C), we assume we can compute $f$ and $g$ using one work tape each. We assign $f$ work tape C and $g$ work tape A.

We cannot use tape B as an input tape for $g$ as this would take too much space, resulting in a non-logspace computation. Instead, we use a \textit{virtual} input tape. This means that the current input position $j$ is stored on work tape B (using a logarithmic amount of space), and in each step we work out $f(x)_{j}$, using work tape C.

All of these components are logspace, meaning our composition function $x$ is also logspace as if $L \leq_{L} L^{\prime}$ then $L^{\prime}\in \mathbf{L} $ implies $L \in \mathbf{L} $.

\subsection{Logspace reduction are polytime}

Let $f$ be a logspace reduction. The bitwise problem is in $\bf L$ and therefore also in $\bf P$.

So, for any $x$, the length of $f(x)$ is polynomially bounded and each bit can be computed in polynomial time, allowing us to compute $f(x)$ in polynomial time (polynomially many steps over polynomially bits).

\subsection{Application: \textbf{P}-completeness}

Just as polytime reductions give a reasonable notion of $\NP$-completeness, so logspace reductions give a reasonable notion of $\mathbf{P} $-completeness.

With $\mathbf{P} $-completeness, we cannot simply look to the degree of the polynomial to determine how \textit{hard} it is, as these are infinite and so $\bf P$ would be the same as $\bf P$-complete. We instead need a different measure.

\begin{definition}($\bf P$-completeness)
A problem is $\bf P$-complete if it is in $\bf P$ and every problem in $\bf P$ logspace-reduces to it.
\end{definition}




\end{document}
